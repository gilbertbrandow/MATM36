\documentclass[12pt]{article}
\usepackage[a4paper,margin=1in]{geometry}
\usepackage{amsmath,amssymb}
\usepackage{enumitem}
\usepackage[a4paper,margin=1in]{geometry}
\usepackage{amsmath,amssymb}
\usepackage{mathtools}
\usepackage{hyperref}
\usepackage{titlesec}
\usepackage{amsthm}
\title{Topology: Extra Problems 1}
\author{Simon Gustafsson}
\date{}
\begin{document}
\maketitle
\section*{Problem 1}
\textbf{(a)}
\begin{proof}
By definition $\varnothing \in \mathcal T_d$ and $\mathbb R \in \mathcal T_d$ since $\mathbb R \setminus \mathbb R = \varnothing$ is finite, hence $\mathcal T_a=\{\varnothing,\mathbb R\}\subset \mathcal T_d$.
Let $U=\mathbb R\setminus\{0\}$; then $\mathbb R\setminus U=\{0\}$ is finite, so $U\in\mathcal T_d$ but $U\notin\mathcal T_a$, hence $\mathcal T_a\subsetneq\mathcal T_d$.

Let $U\in\mathcal T_d$. If $U=\varnothing$ then $U\in\mathcal T_c$. Otherwise $\mathbb R\setminus U$ is finite, hence closed in the standard topology, so $U=(\mathbb R\setminus U)^c$ is open and $U\in\mathcal T_c$. Thus $\mathcal T_d\subset \mathcal T_c$.
However, $(0,1)\in\mathcal T_c$ but $(0,1)\notin\mathcal T_d$ since $\mathbb R\setminus(0,1)$ is infinite, hence $\mathcal T_d\subsetneq\mathcal T_c$.

Since $\mathcal T_b=\mathcal P(\mathbb R)$, every subset of $\mathbb R$ is in $\mathcal T_b$, in particular every $U\in\mathcal T_c$, hence $\mathcal T_c\subset \mathcal T_b$.
Moreover, $\{0\}\in\mathcal T_b$ but $\{0\}$ closed $\implies \{0\}\notin\mathcal T_c \implies \mathcal T_c\subsetneq \mathcal T_b$.
\end{proof}

\textbf{(b)}
\begin{proof}
Since $\mathbb R\setminus U$ finite $\Rightarrow \mathbb R\setminus U$ countable, we have $\mathcal T_d\subset \mathcal T_e$.
Let $U=\mathbb R\setminus\mathbb Z$. Then $\mathbb R\setminus U=\mathbb Z$ is countably infinite, so $U\in\mathcal T_e$, but $U\notin\mathcal T_d$; hence $\mathcal T_d\subsetneq \mathcal T_e$.

Since $\mathcal T_b=\mathcal P(\mathbb R)$, every subset of $\mathbb R$ is in $\mathcal T_b$, in particular every $U\in\mathcal T_e$, hence $\mathcal T_e\subset \mathcal T_b$.
Moreover, $\{0\}\in\mathcal T_b$ but $\{0\}\notin\mathcal T_e$ since $\mathbb R\setminus\{0\}$ is uncountable; therefore $\mathcal T_e\subsetneq \mathcal T_b$.
\end{proof}

\section*{Problem 2}
\textbf{(a)}
\begin{proof}
$d_X\ge 0 \wedge d_Y \ge 0 \implies d \ge 0$. Assume $d((x_1,y_1),(x_2,y_2))=0$. Then
\[
\max\{d_X(x_1,x_2),\,d_Y(y_1,y_2)\}=0,
\]
which implies $d_X(x_1,x_2)=d_Y(y_1,y_2)=0$. Since $d_X$ and $d_Y$ are metrics, this gives $x_1=x_2$ and $y_1=y_2$, hence $(x_1,y_1)=(x_2,y_2)$.
If $(x_1,y_1)=(x_2,y_2)$, then $d_X(x_1,x_2)=d_Y(y_1,y_2)=0$, so $d((x_1,y_1),(x_2,y_2))=0$. Let $(x_i,y_i)\in X\times Y$,
\[
\begin{aligned}
d_X(x_1,x_3)
&\le d_X(x_1,x_2)+d_X(x_2,x_3)
\le d((x_1,y_1),(x_2,y_2))+d((x_2,y_2),(x_3,y_3)), \\
d_Y(y_1,y_3)
&\le d_Y(y_1,y_2)+d_Y(y_2,y_3)
\le d((x_1,y_1),(x_2,y_2))+d((x_2,y_2),(x_3,y_3)), \\
d((x_1,y_1),(x_3,y_3))
&= \max\{d_X(x_1,x_3),\,d_Y(y_1,y_3)\}
\le d((x_1,y_1),(x_2,y_2))+d((x_2,y_2),(x_3,y_3)).
\end{aligned}
\]
\end{proof}

\textbf{(b)}
\begin{proof}
Assume $(x',y')\in B_r^{d}((x,y))$. Then $d((x',y'),(x,y))=\max\{d_X(x',x),\,d_Y(y',y)\}<r$,
so $d_X(x',x)<r \implies x'\in B_r^{d_X}(x)$ and $d_Y(y',y)<r \implies y'\in B_r^{d_Y}(y)$. Therefore $(x',y')\in B_r^{d_X}(x)\times B_r^{d_Y}(y)$.

Assume $(x',y')\in B_r^{d_X}(x)\times B_r^{d_Y}(y)$. Then $d_X(x',x)=r_1<r$ and $d_Y(y',y)=r_2<r$, hence $d((x',y'),(x,y))=\max\{r_1, r_2\}<r \implies (x',y')\in B_r^{d}((x,y))$
\end{proof}

\textbf{(c)}
\begin{proof}
$U$ open $\implies \exists r_1>0 : B^{d_X}_{r_1}(x)\subseteq U$, and since $V$ is open, $\exists r_2>0 : B^{d_Y}_{r_2}(y)\subseteq V$.
Let $\varepsilon=\min\{r_1,r_2\}$. Then
\[
B^{d}_{\varepsilon}((x,y))=B^{d_X}_{\varepsilon}(x)\times B^{d_Y}_{\varepsilon}(y)
\subseteq B^{d_X}_{r_1}(x)\times B^{d_Y}_{r_2}(y)\subseteq U\times V.
\]
Let $(x,y)\in X\times Y$ and $\varepsilon>0$. Set
$
U:=B^{d_X}_{\varepsilon}(x)\subseteq X, V:=B^{d_Y}_{\varepsilon}(y)\subseteq Y.
$
Then $U$ and $V$ are open, $(x,y)\in U\times V$, and by previous
\[
U\times V = B^{d_X}_{\varepsilon}(x)\times B^{d_Y}_{\varepsilon}(y)=B^{d}_{\varepsilon}((x,y)),
\]
hence $(x,y)\in U\times V\subseteq B^{d}_{\varepsilon}((x,y))$.

\end{proof}

\end{document}