\documentclass[12pt]{article}
\usepackage[a4paper,margin=1in]{geometry}
\usepackage{amsmath,amssymb}
\usepackage{enumitem}
\usepackage[a4paper,margin=1in]{geometry}
\usepackage{amsmath,amssymb}
\usepackage{mathtools}
\usepackage{hyperref}
\usepackage{titlesec}
\usepackage{amsthm}
\title{Topology: Exercises 1}
\author{Simon Gustafsson}
\date{}
\begin{document}
\maketitle
\section*{Problem 1}
\begin{proof}
($\Rightarrow$) Assume there exists a bijection $h:A\to B$.
Since $A$ is countable, there exists a set $C$ and a bijection $f:A\to C$,
where $C=\{1,\dots,n\}$ if $A$ is finite and $C=\mathbb N$ if $A$ is countably infinite.
Define $g:=f\circ h^{-1}:B\to C$. Since compositions of bijections are bijections,
$g$ is a bijection. Hence there exisrs a bijection from both $A$ and $B$ to the same set $C$,
so they have the same cardinality.

($\Leftarrow$) Assume $A$ and $B$ have the same cardinality.
Then there exists a set $C \subseteq \mathbb N$ and bijections $f:A\to C$ and $g:B\to C$. Define \(h:=g^{-1}\circ f:A\to B.\)
Since $f$ and $g$ are bijections, $h$ is a bijection. \qedhere
\end{proof}

\section*{Problem 2}
\begin{proof}
Let $k\in\mathbb N$ and choose distinct primes $p_1,\dots,p_k$. Define
\[
f:\mathbb N^k\to\mathbb N,\qquad
(n_1,\dots,n_k)\mapsto p_1^{n_1}\cdots p_k^{n_k}.
\]
Let $x,y\in\mathbb N^k$ and suppose $f(x)=f(y)$. Then
\[
p_1^{x_1}\cdots p_k^{x_k}=p_1^{y_1}\cdots p_k^{y_k}.
\]
By uniqueness of prime factorization, we must have $x_i=y_i$ for each
$i\in \{1,\dots,k\}$. Hence $x=y$, and $f$ is injective.
Since $\mathbb N$ is countable and $\mathbb N^k$ injects into $\mathbb N$,
it follows by Theorem 7.1 that $\mathbb N^k$ is countable.
\end{proof}

\section*{Problem 3}
\begin{proof}
$\mathbb{Q}$ is infinite since the map $f:\mathbb{N}\to\mathbb{Q}$, $n\mapsto n$, is injective.

Define $g:\mathbb{Z}\times(\mathbb{Z}\setminus\{0\})\to\mathbb{Q}$ by
\[
g(m,n)=\frac{m}{n}.
\]
Then $g$ is surjective: for any $q\in\mathbb{Q}$, write $q=\frac{m}{n}$ with
$m\in\mathbb{Z}$ and $n\in\mathbb{Z}\setminus\{0\}$, so $q=g(m,n)$.

Since $\mathbb{Z}$ and $\mathbb{Z}\setminus\{0\}$ are countable,
$\mathbb{Z}\times(\mathbb{Z}\setminus\{0\})$ is countable by Munkres Theorem 7.6.
By Munkres Theorem 7.1, there exists a surjection
$h:\mathbb{N} \to \mathbb{Z}\times(\mathbb{Z}\setminus\{0\})$.
Then $g\circ h:\mathbb{N}\to\mathbb{Q}$ is a surjection, so $\mathbb{Q}$ is countable. Therefore $\mathbb{Q}$ is countably infinite.
\end{proof}

\end{document}