\documentclass[12pt]{article}
\usepackage[a4paper,margin=1in]{geometry}
\usepackage{amsmath,amssymb}
\usepackage{enumitem}
\usepackage[a4paper,margin=1in]{geometry}
\usepackage{amsmath,amssymb}
\usepackage{mathtools}
\usepackage{hyperref}
\usepackage{titlesec}
\usepackage{amsthm}
\title{Topology: Exercises 1}
\author{Simon Gustafsson}
\date{}
\begin{document}
\maketitle
\section*{Problem 1}
\begin{proof}
($\Rightarrow$) Assume there exists a bijection $f:A\to B$.
Since $A$ is countable, there exists a countable set $C$ and a bijection $g:A\to C$,
where $C=\{1,\dots,n\}$ if $A$ is finite and $C=\mathbb N$ if $A$ is countably infinite.
Define $h:=g\circ f^{-1}:B\to C$. Since compositions of bijections are bijections,
$g$ is a bijection. Hence there exists a bijection from both $A$ and $B$ to the same set $C$,
so they have the same cardinality.

($\Leftarrow$) Assume $A$ and $B$ have the same cardinality.
Then there exists a set $C \subseteq \mathbb N$ and bijections $f:A\to C$ and $g:B\to C$. Define \(h:=g^{-1}\circ f:A\to B.\)
Since $f$ and $g$ are bijections, $h$ is a bijection. \qedhere
\end{proof}

\section*{Problem 2}
\begin{proof}
Let $k\in\mathbb N$ and choose distinct primes $p_1,\dots,p_k$. Define
\[
f:\mathbb N^k\to\mathbb N,\qquad
(n_1,\dots,n_k)\mapsto p_1^{n_1}\cdots p_k^{n_k}.
\]
Let $x,y\in\mathbb N^k$ and suppose $f(x)=f(y)$. Then
\[
p_1^{x_1}\cdots p_k^{x_k}=p_1^{y_1}\cdots p_k^{y_k}.
\]
By uniqueness of prime factorization, we must have $x_i=y_i$ for each
$i\in \{1,\dots,k\}$. Hence $x=y$, and $f$ is injective.
It then follows immediately by Theorem 7.1 that $\mathbb N^k$ is countable.
\end{proof}

\section*{Problem 3}
\begin{proof}
Firstly, the trivial map $f:\mathbb{N}\to\mathbb{Q}$ $n\mapsto n$ is injective $\implies$ $\mathbb{Q}$ is infinite.

Next consider 
\[
g:\mathbb{Z}\times\mathbb{Z}^*\to\mathbb{Q} \quad
g(m,n)=\frac{m}{n}.
\]
$\forall q\in\mathbb{Q}$, $\exists m \in \mathbb{Z} \wedge \exists n \in \mathbb{Z}^*$ such that $q=\frac{m}{n} \implies g(m,n) = q$ and hence $g$ surjective.

Since $\mathbb{Z}$ and $\mathbb{Z}^*$ are both countable,
$\mathbb{Z}\times\mathbb{Z}^*$ is countable by Theorem 7.6.
By Theorem 7.1, there exists a surjection
$h:\mathbb{N} \to \mathbb{Z}\times\mathbb{Z}^*$.
Then $g\circ h:\mathbb{N}\to\mathbb{Q}$ is a surjection, so $\mathbb{Q}$ is countable. Therefore $\mathbb{Q}$ is countably infinite.
\end{proof}

\section*{Problem 4}
\textbf{a: }
Label all elements in $A = \{a_1, \cdots, a_n\}$. Let $f: \{0, 1\}^n \to \mathcal{P(A)}$, $(x_1, \cdots, x_n)  \mapsto  \{a_i \in A \ | \ x_i = 1\} $. Let $\mathbf{x}, \mathbf{y} \in \{0, 1\}^n $. Let $f(\mathbf{x}) = f(\mathbf{y}) = \{a_{i_1}, \cdots, a_{i_m}\} \subseteq A$. 
This implies, for each $j \in \{1, \cdots, n\} $:
\[
  x_j=y_j=\left\{
  \begin{array}{@{}ll@{}}
    1, & \text{if}\ j \in \{i_1, \cdots, i_m\} \\
    0, & \text{otherwise}
  \end{array}\right .
\] 
Hence $\mathbf{x} = \mathbf{y}$, and since they were arbitrary, $f$ injective. \\
Let $B = \{a_{i_1}, \cdots,  a_{i_m}\} \subseteq A \implies m \leq n$. Define $\mathbf{x}$ such that $x_j = \mathbf{1}_{\{j \in \{i_1, \cdots, i_m\}\}}$. 
Then $f(\mathbf{x}) = B$, since $B$ arbitrary $f$ surjective and therefore bijective.
\\
\\
\textbf{b: }
Let $A=\{a_1,\dots,a_n\}$. By (a) there is a bijection $f:\{0,1\}^n\to\mathcal P(A)$.
Consider
\[
g:\{0,1\}^n\to\{1,\dots,2^n\},\qquad
(x_1,\dots,x_n) \mapsto 1+\sum_{k=1}^n x_k2^{k-1}.
\]
$g$ is a bijection. Hence $h:=g\circ f^{-1}:\mathcal P(A)\to\{1,\dots,2^n\}$ is a bijection,
so $|\mathcal P(A)|=2^n=2^{|A|}$.
\\
\\
\textbf{c: }
Define
\[
f:\{0,1\}^{\mathbb N}\to\mathcal P(\mathbb N),\qquad
(x_1, x_2 \cdots)\mapsto\{\,k\in\mathbb N \mid x_k=1\,\}.
\]
Let $\mathbf{x},\mathbf{y}\in \{0,1\}^{\mathbb N} : f(\mathbf{x})=f(\mathbf{y}) = C \subseteq \mathbb{N}$. Then 
\[
  x_i= y_i =\left\{
  \begin{array}{@{}ll@{}}
    1, & \text{if}\ i \in C \\
    0, & \text{otherwise}
  \end{array} \implies \mathbf{x}=\mathbf{y}\right.
\]
Hence $f$ injective. Let $A \in \mathcal P(\mathbb N)$, $\exists \mathbf{x}\in\{0,1\}^{\mathbb N} : x_k=\mathbf 1_{\{k\in A\}}$. Then $f(\mathbf{x})=A$., Since $A$ arbitrary, $f$ surjective and therefore also bijective.
\\
\\
\textbf{d: } Suppose for contradiction that $\mathcal P(\mathbb N)$ is countable.
Then there is a list $B_1,B_2,B_3,\dots$ of all subsets of $\mathbb N$.
Define
\[
D=\{\,n\in\mathbb N \mid n\notin B_n\,\}.
\]
Then $D\subseteq\mathbb N$, so $D\in\mathcal P(\mathbb N)$, hence $D=B_m$ for some $m$.
But then
\[
m\in D \iff m\notin B_m \iff m\notin D,
\]
a contradiction. Therefore $\mathcal P(\mathbb N)$ is uncountable.

\end{document}