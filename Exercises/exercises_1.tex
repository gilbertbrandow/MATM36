\documentclass[12pt]{article}
\usepackage[a4paper,margin=1in]{geometry}
\usepackage{amsmath,amssymb}
\usepackage{enumitem}
\usepackage[a4paper,margin=1in]{geometry}
\usepackage{amsmath,amssymb}
\usepackage{mathtools}
\usepackage{hyperref}
\usepackage{titlesec}
\usepackage{amsthm}
\title{Topology: Exercises 1}
\author{Simon Gustafsson}
\date{}
\begin{document}
\maketitle
\section*{Problem 1}
\begin{proof}
($\Rightarrow$) Assume there is a bijection $h:A\to B$.
Since $A$ countable, there exists $f:A\to C \subseteq \mathbb N$. Let
$g:=f\circ h^{-1}:B\to C$. Since composition preserves bijections, $g$ is a bijection. Implying cardinality of $B$ and $A$ determined by $C$.
Hence $A$ and $B$ have the same cardinality.

($\Leftarrow$) Assume $A$ and $B$ have the same cardinality.
Then there exists a set $C \subseteq \mathbb N$ and bijections $f:A\to C$ and $g:B\to C$. Define \(h:=g^{-1}\circ f:A\to B.\)
Since $f$ and $g$ are bijections, $h$ is a bijection. \qedhere
\end{proof}
\end{document}